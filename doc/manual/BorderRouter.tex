\chapter{Border Router}
A SCION border router forward a packet either to an internal server
or to an egress interface based on the packet type.

\section{Data Packet Forwarding}
A SCION data packet is the packet that does not go through any server
(i.e., \CS, \BS, \PS) and carries a series of opaque fields from a
source to a destination (i.e., an end-to-end communication packet).
When an ingress router of an \AD receives a SCION data packet, the
router (1) looks at the current hop (to find out which opaque field
it has to verify), (2) verifies the opaque field, (3) increases the
Hops field by one (viz., Section~\ref{subsec:common-header}) if the
verification passes, and (4) forwards the packet to the egress
interface specified in the opaque field. For opaque field
verification, the router computes a key that corresponds to the
timestamp and expiration time in the packet. As described in
Section~\ref{subsec:key-table-management}, each border router first
refers its Key Table to get the key and then if the table does not
hold a valid key record (i.e., either does not have a record or has a
expired key), the router computes the corresponding key and stores it
to the table.

A series of opaque fields represent an interface-level \AD path from
the \ISDC to a \STUB \AD. Since the ingress interface of an opaque
field is determined by the PCB propagation direction (i.e., an
interface to a provider is always an ingress interface), the
interface needs to be switched

\section{Control Packet Forwarding}
A SCION control packet is the packet that needs to be delivered to
any one of the servers. For example, a PCB needs to be delivered to
the \BS; a certificate request to the \CS, and a path
registration/resolution to the \PS. All control packets do not
include the destination AID field in the SCION header, yet are
distinguished as such by the packet type. Border routers, when they
receive a control packet, write the destination AID with that of the
corresponding server using the up-to-date server locations advertised
by the \CS. Intra-\AD routing/forwarding protocol deliver control
packets to the corresponding servers using the destination AID.

{\bf PCB propagation: } The ingress (border) router of an \AD, when
it receives a PCB from its parent \AD, writes the Current IF field of
the header (viz., Figure~\ref{fig:hdr-common}) with the interface ID
at which at PCB arrive, in order to help the \BS to generate a opaque
field (note that ingress interface ID is necessary for MAC
generation).
 
\section{Bootstrapping}
A border router, on its startup, register itself to the \BS so as to
receive PCBs from the \BS. This registration would be invalidated if
the \BS receives an PCB propagation error message. \soobum{We have to
define error messages for troubleshooting and maintenance; e.g., PCB
propagation error.} The router retrieves the current and prior master
OFG keys from the \CS. Since an opaque field is valid for a day, the
Key Table entries are generated either by the current or by the
previous master OFG key. Hence, the router is able to construct its
Key Table with those two master OFG keys. And, the border router
should get the server locations from either any neighboring router or
directly from the \CS. The border router specifies its request in the
SCION header, and an internal router of the \AD responds this request
or forwards it to the \CS. We note that the master OFG response is
encrypted includes the MAC generated by the \CS; and the server
location response includes the \CS's signature.
