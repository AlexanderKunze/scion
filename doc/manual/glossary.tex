% Glossary and Abbreviation List.

% Responsibility: Raphael // Please contact me in case of
% questions or inconsistencies.


%%% Syntax and usage %%%%%%%%%%%%%%%%%%%%%%%%%%%%%%%%%%%%%%

%    \gitem{Title}{Description}

% Creates an entry of the form Title --> Description.
% This list of items is automatically sorted by title.
% When inserting an entry, don't care about its position.

% When you introduce an abbreviated entry, e.g., Core Beacon
% Server (CBS), then please add below your \gitem entry an
% abbreviation entry as follows:

%    \abbrev{CBS}{Core Beacon Server}

% This entry will appear in the abbreviation list (behind
% the glossary). Moreover, the abbreviation will appear in
% the index.


%%% How to create manual index entries? %%%%%%%%%%%%%%%%%%%

% If you want to add a manual entry to the index, please
% use the additional optional argument as follows:

%    \gitem[Keyword]{Title}{Description}

% The keyword string will appear in the index and point to
% to the page where the glossary entry appears.

% As mentioned above, the \abbrev command also adds an index
% entry for the abbreviation. You can, however, specify an
% individual index entry as follows.

%    \abbrev[Keyword]{CBS}{Core Beacon Server}

% In this case, the index will contain the entry Keyword,
% instead of the long string ("Core Beacon Server" here).


\newcommand\go{$\to$}


\begin{SCIONglossary}


% Paths.

\gitem[half-path]{Half-Path}{Up-path from an end entity to the ISD
core, or down-path from the ISD core to an end entity.}

\gitem[inter-domain path]{Inter-Domain Path}{Path between two ADs
that are located in two different ISD cores.}

\gitem[end-to-end path]{End-to-End Path}{Path between two end
entities, typically a combination of three paths: an up-path, an
inter-domain path, and a down-path.}

\gitem[static path]{Static Path}{Long-lived, low-bandwidth
reservation on top of half-paths. Static paths are initiated by leaf
ADs and are used in SIBRA for two purposes: (1) as a building block
for priority paths: to guarantee availability during connection
setup, and to perform weighted bandwidth reservation, and (2) to
provide communication guarantees for low-bandwidth traffic, e.g.
control traffic. See also \go~Priority Path.}

\gitem[priority path]{Priority Path}{Short-lived, high-bandwidth
reservation on top of end-to-end paths, i.e., up-path, inter-domain
path, and down-path. Priority paths are initiated by end hosts in
SIBRA. They are only valid on the order of tens of seconds, and thus
needs to be continuously renewed. See also \go~Static Path.}




% Servers.

%% Path Server.

\gitem{Path Server}{Service within an \go~Isolation
Domain where end hosts register their up- and down-paths. End hosts
can query path servers in order to obtain valid paths do a
destination. \\ For more information, please check the entries for
\go~Core Path Server (CPS) and \go~Local Path Server (LPS)}

\gitem[Core Path Server]{Core Path Server (CPS)}{Path Server located in the ISD core,
more precisely within a core AD. Non-core ADs within the ISD register
their intra-ISD down-paths, the CPS stores the paths, and resolves
inter-ISD paths. More precisely, each \go~Beacon Server (BS) within the
same ISD registers up to $k$ down-paths with the CPS. The CPS
receives from a Local Path Server (LPS) as input a SCION destination
address and returns up to $k$ down-paths to the destination AD. If
the destination AD resides within a different ISD, the CPS requests
the down-paths from the remote (destination) ISD's core path server
before returning these paths to the LPS. To reduce path resolution
latency, a CPS can cache down-paths received from a CPS in a
different ISD. A CPS can be replicated within an AD to increase
availability. Additionally, all core path servers within a single ISD
run a consistency protocol to ensure a consistent view of ISD-local
paths.} \abbrev{CPS}{Core Path Server}

\gitem[Local Path Server]{Local Path Server (LPS)}{Service within a non-core A that
retrieves from \go~Local Beacon Servers (LBS) a path from itself to a
\go~SCION address. The LPS receives (from a local client) as input a
SCION destination address, and returns up to $k$ up-paths and up to
$k$ down-paths between the AD where the LPS resides and the
destination's AD. Note that the LPS cannot independently resolve the
full path across ISDs, so queries to paths outside of the current ISD
are forwarded to \go~Core Path Servers (CPS). LPSes can cache
down-paths received from a core path server to reduce path resolution
latency. ADs may choose to replicate LPS functionality across
multiple devices increase availability.}

\abbrev{LPS}{Local Path Server}




% Beacon Server.

\gitem[Beacon Server]{Beacon Server (BS)}{Service that
enables path exploration within an ISD. The ISD core beacon
server starts the propagation of Path Construction Beacons
(PCBs) to construct a half-path from the ISD Core to
endpoint ADs. Upon receiving a PCB, a BS can choose to
register the learned path with the Core Path Server (CPS)
and its Local Path Server (LPS). The BS then propagates the
PCB to its down-stream and peering ADs and appends for each
of these ADs a corresponding routing decision, known as an
\go~Opaque Field (OF). A BS located in an endpoint AD only
registers paths with the LPS and the CPS, since PCBs are
never propagated up-stream. \\ For additional information,
please check the entries for \go~Core Beacon Server (CBS)
and \go~Local Beacon Server (LBS).}

\abbrev{BS}{Beacon Server}


\gitem[Core Beacon Server]{Core Beacon Server (CBS)}{Beacon Server
(BS) located in the ISD core, more precisely within a core AD. A CBS
initiates the propagation of Path Construction Beacons (PCBs). \\ For
more information, please check the entry for \go~Beacon Server (BS).}

\abbrev{CBS}{Core Beacon Server}


\gitem[Local Beacon Server]{Local Beacon Server (LBS)}{Beacon Server (BS) located in a
non-core AD that receives all Path Construction Beacons (PCBs) from
the provider ADs. At every propagation period, an LBS performs two
steps: (a) it selects $k$ PCBs and registers them as down-paths at
the Core Path Server (CPS), and (b) it selects $k$ PCBs and registers
them as up-paths at the Local Path Server (LPS). Selection criteria
for down- and up-paths may differ. \\ For additional information,
please check the entry for \go~Beacon Server (BS).}

\abbrev{LBS}{Local Beacon Server}




% Log Server.

\gitem[log server]{Log server}{(Also public log, or integrity log
server) Public server that monitors \go~CA operations (most commonly
issued or revoked certificates) to ensure that CAs are behaving
correctly. A log server answers queries regarding CA actions.
Operations are most commonly logged using \go~Merkle hash trees in a
time-ordered fashion, allowing log servers to show consistency (no
events have been retroactively changed).}





% ISDs.

\gitem[Isolation Domain]{Isolation Domain (ISD)}{Hierarchical
grouping of networks under a common organizational domain. Networks
within an ISD should share a common jurisdiction. Each ISD designates
a set of core Autonomous Domains (ADs) that provide service to
subscriber ADs.} \abbrev{ISD}{Isolation Domain}

\gitem[ISD core]{ISD Core}{Set of ADs within an \go~Isolation Domain
responsible for handling inter-ISD routing, quality of service within
an ISD, and accountability management (e.g., disabling connectivity
for misbehaving ADs).}



% Opaque Fields.

\gitem[Opaque Field]{Opaque Field (OF)}{Opaque fields are often used in
the SCION packet header. They serve different purposes,
such as: routing, authenticity verification, and parsing of
the SCION packet itself. Please check the individual
entries: \go~Hop Opaque Field, \go~Info Opaque Field,
\go~ROT Field, \go~Support Signature Field, \go~Support
Peer Field, \go~Support PCB Field.}

\abbrev{OF}{Opaque Field}

\gitem{Hop Opaque Field}{Opaque field
used to identify the packet sender, i.e. by storing the
sender interfaces.}

\gitem[Info Opaque Field]{Info Opaque Field}{Opaque field
used to store further information about the path and the
packet itself, such as: Isolation Domain ID and number of
hops in the path.}

\gitem[ROT Field]{ROT Field}{Opaque field used to store
information about the Root Of Trust file used (i.e. ROT
version).}

\gitem[Support Signature Field]{Support Signature
Field}{Opaque field used to parse the packet and verify the
packet’s signature.}

\gitem{Support Peer Field}{Opaque field used to store information
about a peer link’s bandwidth (i.e. bandwidth allocated, class).}

\gitem{Support PCB Field}{Opaque field used to store information
about a parent-child link’s bandwidth (i.e. bandwidth allocated,
bandwidth reserved).}






% Misc.

\gitem[Path Construction Beacon]{Path Construction Beacon (PCB)}{to
be done!} \abbrev{PCB}{Path Construction Beacon}

\gitem[Reservation Token]{Reservation Token (RT)}{Cryptographically authenticated token
that enables the usage of SIBRA bandwidth reservations. RTs are
enriched opaque fields that encode the amount of bandwidth and the
validity of an AD's bandwidth reservation for a specific flow.}
\abbrev{RT}{Reservation Token}

\gitem[DRKey protocol]{DRKey Protocol}{Protocol to enable routers
$R_i$ on a source-specified path $\langle S, R_1, \allowbreak R_2,
\ldots, D\rangle$ to set up on-the-fly shared keys with source $S$
and destination $D$. DRKey protocols enable routers to re-derive keys
on the fly when needed, thus avoiding the need of per-flow state on
routers.}

\gitem[DRKey session]{DRKey Session}{Session that identifies the
packets sent by a source $S$ to a destination $D$ on the same path,
during a certain period of time. A DRKey session is typically
associated with a flow that travels the same network path, such as a
TCP flow in SCION.}

\gitem[Retroactive-DRKey]{Retroactive-DRKey}{DRKey protocol that
enables entities on the path to set up shared keys at any time after
the first packet in a DRKey session reaches the destination.
Retroactive-DRKey hides the source's intention to set up shared keys
with the routers on the path, preventing coward attacks.}

\gitem[SCION address]{SCION Address}{Network-level address of a device
using SCION. A SCION address takes the form
\texttt{ISD:AD:client\_addr}. \texttt{ISD} is a unique identifier
among ISDs, \texttt{AD} is a unique identifier within an ISD, and
\texttt{client\_addr} is some network-level identifier within an AD.
The \texttt{client\_addr} must be unique within an AD, but may be,
e.g., an IPv4 address, IPv6 address, or other local network address
managed by the corresponding AD.}

% Authentication / Certification.

\gitem[certificate authority]{Certificate/Certification Authority
(CA)}{(Also Certificate Agency in AKI) Issues public-key certificates
that bind information (most often a name) to a public key. CAs have
the responsibility of ensuring that the information-to-public-key
binding is correct. They also possess one or more key pairs used to
sign and verify certificates.} \abbrev{CA}{Certificate Authority}

\gitem[Merkle hash tree]{Merkle hash tree (MHT)}{Tree-based data
structure in which leaf nodes hold data objects and non-leaf nodes
hold the hash of their child nodes. MHTs can prove that a data object
is present in the tree using a logarithmic number of nodes, and if
the data is ordered in the leaf nodes, the MHT can also prove that a
data object is absent from the tree.} \abbrev{MHT}{Merkle hash tree}

\gitem[validator]{Validator}{(Also monitor, or auditor in CT) A party
that monitors \go~log servers to ensure that they are behaving
correctly. Validators determine correct behavior by downloading all
or part of the log server's database and performing consistency
checks on them, by verifying individual proofs of presence or
absence, and by watching for suspicious certificates.}

\end{SCIONglossary}

